%iffalse
\let\negmedspace\undefined
\let\negthickspace\undefined
\documentclass[journal,12pt,twocolumn]{IEEEtran}
\usepackage{cite}
\usepackage{amsmath,amssymb,amsfonts,amsthm}
\usepackage{algorithmic}
\usepackage{graphicx}
\usepackage{textcomp}
\usepackage{xcolor}
\usepackage{txfonts}
\usepackage{listings}
\usepackage{multicol}
\usepackage{enumitem}
\usepackage{mathtools}
\usepackage{gensymb}
\usepackage{comment}
\usepackage[breaklinks=true]{hyperref}
\usepackage{tkz-euclide} 
\usepackage{listings}
\usepackage{gvv}                                        
%\def\inputGnumericTable{}                                 
\usepackage[latin1]{inputenc}                                
\usepackage{color}                                            
\usepackage{array}                                            
\usepackage{longtable}                                       
\usepackage{calc}                                             
\usepackage{multirow}                                         
\usepackage{hhline}                                           
\usepackage{ifthen}                                           
\usepackage{lscape}
\usepackage{tabularx}
\usepackage{array}
\usepackage{float}


\newtheorem{theorem}{Theorem}[section]
\newtheorem{problem}{Problem}
\newtheorem{proposition}{Proposition}[section]
\newtheorem{lemma}{Lemma}[section]
\newtheorem{corollary}[theorem]{Corollary}
\newtheorem{example}{Example}[section]
\newtheorem{definition}[problem]{Definition}
\newcommand{\BEQA}{\begin{eqnarray}}
\newcommand{\EEQA}{\end{eqnarray}}
\newcommand{\define}{\stackrel{\triangle}{=}}
\theoremstyle{remark}
\newtheorem{rem}{Remark}

% Marks the beginning of the document
\begin{document}
\renewcommand{\thefigure}{\theenumi}
\renewcommand{\thetable}{\theenumi}
\bibliographystyle{IEEEtran}
\vspace{3cm}

\title{Assignment-1 (extra) }
\author{AI24BTECH11036- Shreedhanvi Yadlapally}
\maketitle
\newpage
\bigskip
\section{Subjective Problems}
\begin{enumerate}

	\item Tangent at a point $P_{1}$ {other than \brak{0, 0}} on the curve $y=x^{3}$ meets the curve again at $P_{2}$. The tangent at $P_{2}$ meets the curve again at $P_{1}$, and so on. Show that the abscissae of $P_{1}, P_{2}, P_{3} \dots P_{n}$, form a G.P. Also find the ratio. $\frac{\sbrak{area(\Delta P_{1}P_{2}P_{3})}}{\sbrak{area(\Delta P_{2}P_{3}P_{4})}}$

	\hfill{(1993 - 5 Marks)}

\item A line through $A \brak{5, 4}$ meets the line $x+3y+2=0$, $2x+y+4=0$ and $x-y-5=0$ at the points $B$, $C$ and $D$ respectively. If $\brak{15/AB}^{2}+\brak{10/AC}^{2}-\brak{6/AD}^{2}$, find the equation of the line.

	\hfill{(1993 - 5 Marks)}

\item A rectangle $PQRS$ has it side $PQ$ parallel to the line $y=mx$ 
and vertices $P$, $Q$ and $S$ on the lines $y=a$, $x=b$ and $x=-b$, 
respectively. Find the locus of the vertex $R$.
	
		\hfill{(1996 - 2 Marks)}

\item Using co-ordinate geometry, prove that the three altitudes of any triangle are concurrent.

	\hfill{(1998 - 8 Marks)}

\item For points $P=\brak{x_{1}, y_{1}}$ and $Q=\brak{x_{2}, y_{2}}$ of the co-ordn=inate plane, a new distance $d\brak{P, Q}$ is defined by $d\brak{P, Q}=\abs{x_{1}-x_{2}}+\abs{y_{1}-y_{2}}$. Let $O=\brak{0, 0}$ and $A=\brak{3, 2}$. Prove that the set of points in the first quadrant which are equidistant (with respect to the new distance) from $O$ and $A$ consists of the union of a line segment of finite length and an infinite ray. Sketch this net in a labelled diagram.

	\hfill{(2000 - 10 Marks)}

\item Let $ABC$ and $PQR$ be any two triangles in the same plane.
Assume that the perpendiculars from the points $A, B, C$ to 
the sides $QR, RP, PQ$ respectively are concurrent. Using vector methods or otherwise, prove that the perpendiculars from $P, Q, R$ to $BC, CA, AB$ respectively are also concurrent. 

	\hfill{(2000 - 10 Marks)}

\item Let $a, b, c$ be real numbers with $a^{2}+b^{2}+c^{2}=1$. Show that the equation $\mydet{
ax-by-c     &bx+ay       &cx+a      \\
bx+ay       &-ax+by-c   &cy+b       \\
cx+a         &cy+b         &-ax-by+c \\} =0$
represents a straight line.

\hfill{(2001 - 6 Marks)}

\item A straight line $L$, through the origin meets the lines $x+y=1$ and $x+y=3$ at $P$ and $Q$ respectively. Through $P$ and $Q$ two 
straight lines $L_{1}$, and $L_{2}$ are drawn, parallel to $2x-y=5$ and $3x+y=5$ respectively. Lines $L_{1}$ and $L_{2}$ intersect at $R$. Show 
that the locus of $R$, as $L$, varies, is a straight line. 
\hfill{(2002 - 5 Marks)}

\item A straight line $L$ with negative slope passes through the 
point \brak{8, 2} and cuts the positive coordinate axes at points 
$P$ and $Q$. Find the absolute minimum value of $OP + OQ$, as $L$ 
varies, where $O$ is the origin.

\hfill{(2002 - 5 Marks)}

\item The area of the triangle formed by the intersection of a line 
parallel to x-axis and passing through $P\brak{h, k}$ with the lines 
$y=x$ and $x+y=2$ is $4h^{2}$, Find the locus of the point P.

\hfill{(2005 - 2 Marks)}

\end{enumerate}

\section{Assertion and Reason Type Questions} 
\begin{enumerate}
\item Lines $L_{1}: y-x=0$ and $L_{2}: 2x+y=0$ intersect the line $L_{3}: y+2=0$ at $P$ and $Q$, respectively. The bisector of the acute 
angle between $L_{1}$ and $L_{2}$ intersects $L_{3}$ at $R$.\\
\textbf{STATEMENT-1 :} The ratio $PR:RQ$ equals $2\sqrt{2}:\sqrt{5}$.\\
\textbf{STATEMENT-2 :} In any triangle, bisector of an angle divides the triangle into two similar triangles.

\hfill{(2007 - 3 Marks)}
   \begin{enumerate}[label=(\alph*)]
   \item Statement-1 is True, Statement-2 is True Statement-2 
is not a correct explanation for Statement-1 
   \item Statement-1 is True, Statement-2 is True; Statement-2 
is NOT a correct explanation for Statement-1 
   \item Statement-I is True, Statement-2 is False
   \item Statement-1 is False, Statement-2 is True. 
   \end{enumerate}
\end{enumerate}

\section{Integer Value Correct Type}
\begin{enumerate}
\item For a point $P$ in the plane, let $d_{1}(P)$ and $d_{2}(P)$ be the 
distance of the point P from the lines $x-y=0$ and $x+y =0$ 
respectively. The area of the region $R$ consisting of all points 
$P$ lying in the first quadrant of the plane and satisfying
$2\leq d_{1}(P)+d_{2}(P) \leq4$, is

		\hfill{(JEE Adv. 2014)}
\end{enumerate}
\end{document}
   
