\let\negmedspace\undefined
\let\negthickspace\undefined
\documentclass[journal]{IEEEtran}
\usepackage[a5paper, margin=10mm, onecolumn]{geometry}
%\usepackage{lmodern} % Ensure lmodern is loaded for pdflatex
\usepackage{tfrupee} % Include tfrupee package

\setlength{\headheight}{1cm} % Set the height of the header box
\setlength{\headsep}{0mm}     % Set the distance between the header box and the top of the text

\usepackage{gvv-book}
\usepackage{gvv}
\usepackage{cite}
\usepackage{amsmath,amssymb,amsfonts,amsthm}
\usepackage{algorithmic}
\usepackage{graphicx}
\usepackage{textcomp}
\usepackage{xcolor}
\usepackage{txfonts}
\usepackage{listings}
\usepackage{enumitem}
\usepackage{mathtools}
\usepackage{gensymb}
\usepackage{comment}
\usepackage[breaklinks=true]{hyperref}
\usepackage{tkz-euclide} 
\usepackage{listings}
% \usepackage{gvv}                                        
\def\inputGnumericTable{}                                 
\usepackage[latin1]{inputenc}                                
\usepackage{color}                                            
\usepackage{array}                                            
\usepackage{longtable}                                       
\usepackage{calc}                                             
\usepackage{multirow}                                         
\usepackage{hhline}                                           
\usepackage{ifthen}                                           
\usepackage{lscape}
\begin{document}

\bibliographystyle{IEEEtran}
\vspace{3cm}

\title{1.6.6}
\author{AI24BTECH11036 - Yadlapally Shreedhanvi}
% \maketitle
% \newpage
% \bigskip
{\let\newpage\relax\maketitle}

\renewcommand{\thefigure}{\theenumi}
\renewcommand{\thetable}{\theenumi}
\setlength{\intextsep}{10pt} % Space between text and floats


\numberwithin{equation}{enumi}
\numberwithin{figure}{enumi}
\renewcommand{\thetable}{\theenumi}


\textbf{Question}:In each of the following, find the value of k, for which the points are collinear.\\
a) (7, -2), (5, 1), (3, k)\\
b) (8, 1), (k, -4), (2, -5)\\ \\

\textbf{Solution: }

\begin{table}[h!]    
  \centering
  \begin{tabular}[12pt]{ |c| c|}
    \hline
    \textbf{Variable} & \textbf{Description}\\ 
    \hline
	${v_{m}}$ & Speed of the man in still water\\
    \hline 
	${v_{r}}$ & Speed of the river flow\\
    \hline
        $w$ & Width of the river\\
    \hline
	$t$ & Time taken to cross the river\\
    \hline
	$d$ & Distance drifted downstream\\
    \hline
\end{tabular}

  \caption{Co-ordinates}
  \label{tab1.6.6.1}
\end{table}
The rank-nullity theorem states that for any linear transformation, ($n$ is the dimension of the space) $\vec{M}$, 
\begin{align}
	\vec{M}=\myvec{\vec{A_{2}-A_{1}} & \vec{A_{3}-A_{1}} & \dots & \vec{A_{m}-A_{1}}} 
\end{align}

\begin{align}
\text{rank}\brak{\vec{M}}+\text{nullity}\brak{\vec{M}}=n
	\\
\text{rank}\brak{\vec{M}}=1
\end{align}

Consider a set of $m$ points $\vec{A_1,A_2,A_3\dots,A_m}$ in an $n$-dimensional. These points are said to be collinear if they all lie on a single straight line. For collinearity, the vectors formed by subtracting one point from another must be linearly dependent. All the vectors are scalar multiples of each other so the rank must be 1. \\
For the points $\vec{A}$, $\vec{B}$  and  $\vec{C}$ to be collinear,
\begin{align}
	\text{rank} \myvec{ \vec{B-A} & \vec{C-A}}=1
\end{align}
In case (a)
\begin{align*}
\myvec{ \vec{B-A} & \vec{C-A}}=\\
\myvec{ -2 & -4 \\ 3 & k+2} 
\xleftrightarrow[]{R_2 \leftarrow {3R_1+2R_2}}
\myvec{-2 & -4 \\ 0 & -8+2k} \\
\text{Since the rank of the above matrix should be 1,}
-8+2k=0\\
\therefore k=4.
\end{align*}
In case (b)
\begin{align*}
\myvec{ \vec{B-A} & \vec{C-A}}=\\
\myvec{ k-8 & -6 \\ -5 & -6} 
\xleftrightarrow[]{R_2 \leftarrow {R_1-R_2}}
\myvec{k-8 & -6 \\ k-3 & 0} \\
\text{Since the rank of the above matrix should be 1,}
k-3=0\\
\therefore k=3.
\end{align*}







\begin{figure}[h!]
   \centering
   \includegraphics[width=0.7\linewidth]{figs/Figure_1.png}
   \caption{Plots of Lines}
   \label{plot}
\end{figure}
\end{document}  
\end{document}


