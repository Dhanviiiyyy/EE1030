%iffalse
\let\negmedspace\undefined
\let\negthickspace\undefined
\documentclass[journal,12pt,onecolumn]{IEEEtran}
\usepackage{cite}
\usepackage{amsmath,amssymb,amsfonts,amsthm}
\usepackage{algorithmic}
\usepackage{graphicx}
\usepackage{textcomp}
\usepackage{xcolor}
\usepackage{txfonts}
\usepackage{listings}
\usepackage{multicol}
\usepackage{enumitem}
\usepackage{mathtools}
\usepackage{gensymb}
\usepackage{comment}
\usepackage[breaklinks=true]{hyperref}
\usepackage{tkz-euclide} 
\usepackage{listings}
\usepackage{gvv}                                        
%\def\inputGnumericTable{}                                 
\usepackage[latin1]{inputenc}                                
\usepackage{color}                                            
\usepackage{array}                                            
\usepackage{longtable}                                       
\usepackage{calc}                                             
\usepackage{multirow}                                         
\usepackage{hhline}                                           
\usepackage{ifthen}                                           
\usepackage{lscape}
\usepackage{tabularx}
\usepackage{array}
\usepackage{float}


\newtheorem{theorem}{Theorem}[section]
\newtheorem{problem}{Problem}
\newtheorem{proposition}{Proposition}[section]
\newtheorem{lemma}{Lemma}[section]
\newtheorem{corollary}[theorem]{Corollary}
\newtheorem{example}{Example}[section]
\newtheorem{definition}[problem]{Definition}
\newcommand{\BEQA}{\begin{eqnarray}}
\newcommand{\EEQA}{\end{eqnarray}}
\newcommand{\define}{\stackrel{\triangle}{=}}
\theoremstyle{remark}
\newtheorem{rem}{Remark}

% Marks the beginning of the document
\begin{document}
\bibliographystyle{IEEEtran}
\vspace{3cm}

\title{Assignment-6}
\author{AI24BTECH11036- Shreedhanvi Yadlapally}
\maketitle

\bigskip
\renewcommand{\thefigure}{\theenumi}
\renewcommand{\thetable}{\theenumi}
\section{MCQ - 2 marks}
\begin{enumerate}

	\item Let $u\brak{x,y}$ be the real part of an entire function $f\brak{z}=u\brak{x,y}+iv\brak{x,y}$ for for $z=x+iy\in \mathbb{C}$. If $C$ is the positively oriented boundary of a rectangular region $R$ in $\mathbb{R}^{2}$, then $\oint_{C}\sbrak{\frac{\partial u}{\partial y}dx-\frac{\partial u}{\partial x}dy}=$
	\begin{multicols}{4}
	\begin{enumerate}
		\item 1
		\item 0
		\item 2$\pi$
		\item $\pi$
	\end{enumerate}
        \end{multicols}

	\item Let $\phi :\sbrak{0,1}\to \mathbb{R}$ be three times continuously differentiable. suppose that the iterates defined by $x_{n+1}=\phi \brak{x_{n}}, n\geq 0$ converge to the fixed point $\xi$ of $\phi$. If the order of convergence is three then
	\begin{multicols}{2}
	\begin{enumerate}
		\item $\phi \prime \brak{\xi}=0$, $\phi \prime \prime \brak{\xi}=0$
		\item $\phi \prime \brak{\xi}\neq 0$, $\phi \prime \prime \brak{\xi}=0$
		\item $\phi \prime \brak{\xi}=0$, $\phi \prime \prime \brak{\xi}\neq 0$ 
		\item $\phi \prime \brak{\xi}\neq 0$, $\phi \prime \prime \brak{\xi}\neq 0$
	\end{enumerate}
	\end{multicols}

	\item Let $f: \sbrak{0,2}\to \mathbb{R}$ be a twice continuously differentiable function. If $\int_{0}^{2} dx=2f\brak{1}$, then the error in the aproximation is
	\begin{multicols}{2}
	\begin{enumerate}
		\item $\frac{f'\brak{\xi}}{12}$ for some $\xi \in \brak{0,2}$
		\item $\frac{f'\brak{\xi}}{2}$ for some $\xi \in \brak{0,2}$
 		\item $\frac{f''\brak{\xi}}{3}$ for some $\xi \in \brak{0,2}$
		\item $\frac{f''\brak{\xi}}{6}$ for some $\xi \in \brak{0,2}$
	\end{enumerate}
	\end{multicols}

	\item For a fixed $t \in \mathbb{R}$, consider the linear programming problem:
\begin{align*}
\text{Maximize } z=3x+4y \\
\text{subject to } x+y \leq 100 \\
x+3y \leq t \\
\text{and } x \geq 0, y\geq 0
\end{align*}
The maximium value of $z$ is 400 for $t=$ 
	\begin{multicols}{4}
	\begin{enumerate}
		\item 50
		\item 100
		\item 200
		\item 300
	\end{enumerate}
	\end{multicols}

\item The minimum value of $z=2x_{1}-x_{2}+x_{3}-5x_{4}+22x_{5}$ subject to
\begin{align*}
	x_{1}-2x_{4}+x_{5}=6\\
	x_{2}+x_{4}-4x_{5}=3\\
	x_{3}+3x_{4}+2x_{5}=10\\
\end{align*}
is
	\begin{multicols}{4}
	\begin{enumerate}
		\item 28
		\item 19
		\item 10
		\item 9
	\end{enumerate}
	\end{multicols}
	
	\item Using the Hungarian method, the optimal value of the assignment problem whose cost matrix is given by 
%insert table
		\begin{table}[ht]
		\centering
		\begin{tabular}[12pt]{ |c| c|}
    \hline
    \textbf{Label} & \textbf{Co-ordinate}\\ 
    \hline
	\multicolumn{2}{|c|}{\textbf{Case (a)}}\\
	\hline
    $\vec{A}$ & $\brak{7, -2}$ \\
    \hline 
    $\vec{B}$ & $\brak{5, 1}$ \\
    \hline
    $\vec{C}$ & $\brak{3, k}$ \\
    \hline
	\multicolumn{2}{|c|} {\textbf{Case (b)}}\\
	\hline
    $\vec{A}$ & $\brak{8, 1}$ \\
    \hline 
    $\vec{B}$ & $\brak{k, -4}$ \\
    \hline
    $\vec{C}$ & $\brak{2, -5}$ \\
    \hline 
    \end{tabular}

		\end{table}
	\begin{multicols}{4}
	\begin{enumerate}
		\item 29
		\item 52
		\item 26
		\item 44
	\end{enumerate}
	\end{multicols}

	\item Which of the following sequence $\cbrak{f_{n}}_{n=1}^{\infty}$ of functions does NOT converge uniformly on $\sbrak[0,1]$?
	\begin{multicols}{2}
	\begin{enumerate}
		\item $f_{n}\brak{x}=\frac{e^{-x}}{n}$
		\item $f_{n}\brak{x}=\brak{1-x}^{n}$
		\item $f_{n}\brak{x}=\frac{x^{2}+nx}{n}$
		\item $f_{n}\brak{x}=\frac{\sin\brak{nx+n}}{n}$
	\end{enumerate}
	\end{multicols}

	\item Let $E=\cbrak{\brak{x,y}\in \mathbb{R}^{2} : 0<x<y}$. Then $\iint_{E} ye^{-\brak{x+y}}dxdy=$
	\begin{multicols}{2}
	\begin{enumerate}
		\item $\frac{1}{4}$
		\item $\frac{3}{2}$
		\item $\frac{4}{3}$
		\item $\frac{3}{4}$
	\end{enumerate}
	\end{multicols}

	\item Let $f_{n}\brak{x}=\frac{1}{n} \sum _{k=0}^{n} \sqrt{k\brak{n-k}} \binom{n}{k} x^{k} \brak{1-x}^{n-k}$ for $x\in \sbrak{0,1}$, $n=1, 2, \dots$. If $\lim _{n\to \infty} f_{n}\brak{x}=f\brak{x}$ for $x\in \sbrak{0,1}$, then the maximum value of $f\brak{x}$ on $\sbrak{0,1}$ is
	\begin{multicols}{4}
	\begin{enumerate}
		\item 1
		\item $\frac{1}{2}$
		\item $\frac{1}{3}$
		\item $\frac{1}{4}$
	\end{enumerate}
	\end{multicols}

	\item Let $f : \brak{c_{00}, ||\cdot||_{1}} \to \mathbb{C}$ be a non-zero continuous linear function. The numberof Hahn-Banach extensions of $f$ to $\brak{l^{1}, || \cdot ||_{1}}$ is 
	\begin{multicols}{2}
	\begin{enumerate}
		\item one
		\item two
		\item three
		\item infinite
	\end{enumerate}
	\end{multicols}

	\item If $I : \brak{l^{1}, || \cdot ||_{2}} \to \brak{l^{1}, || \cdot ||_{1}}$ is the identity map, then
	\begin{enumerate}
		\item both $I$ and $I^{-1}$ are continuous
		\item $I$ is continuous but $I^{-1}$ is NOT continuous
		\item $I^{-1}$ is continuous but $I$ is NOT continuous
		\item neither $I$ nor $I^{-1}$ is continuous
	\end{enumerate}

\item Consider the topology $\tau = \cbrak{G \subseteq \mathbb{R} : \mathbb{R} | G \text{ is compact in } \brak{\mathbb{R}, \tau _{u}}} \cup \cbrak{\phi, \mathbb{R}}$ on $\mathbb{R}$, where $\tau_{u}$ is the usual topology on $\mathbb{R}$ and $\phi$ is the empty set. Then $\brak{\mathbb{R}, \tau}$ is
	\begin{enumerate}
		\item  a connected Hausdorff space
		\item connected but NOT Hausdorff
		\item Hausdorff but NOT connected
		\item neither connected nor Hausdorff
	\end{enumerate}


\end{enumerate}
\end{document}









